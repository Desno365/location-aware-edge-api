\chapter{Preliminaries and State of the Art}
\label{ch:preliminaries_and_sota}

\section{Preliminary notions}
\label{sec:preliminaries}

\begin{table}[!ht]
\centering
\begin{tabular}{c l} \hline
\textbf{Notation}&\textbf{Description} \\ \hline
$G$&Graph\\
$V$&set of nodes of $G$\\
$E$&set of edges of $G$\\
$W$&set of weights corresponding to each edge in $E$\\
$w_{u,v}$&weight of edge $(u,v)$\\
$n$&$|V|$, number of nodes\\
$m$&$|E|$, number of edges\\
\hline
\end{tabular}
\caption{Graph notation.}
\label{tab:notation}
\end{table}

``In this section, we introduce the preliminary notions at the base of our study. We start by briefly introducing the problem, and then we provide the necessary concepts and the notation used."

You may insert a subsection for each of the most relevant features of your problem. You can add some reference if needed, but just to explain the problem. The references with the solutions of the problem should be put in the next section.

You can keep a notation table for the notation used in this chapter as \autoref{tab:notation}. Everything inside the notation table must be written at least once inside this chapter. You can put an extended notation for the whole thesis in the appendix.

It is likely that you have to present definitions, theorems or propositions. We suggests to use the environments provided by the template. You can find the guide in the LaTeX suggestions chapter.

\section{State of the Art}
\label{sec:sota}

In this section, we survey the most relevant works related to the argument of your thesis. If you face a problem that has more than one macro-topic, you may choose to add a subsection for each of these topics (better no more than 2-3), like \emph{Related works on Topic 1}, etc.

List the works in chronological order and cite only the most important and pertinent ones, avoid 100 citations for a master thesis.