\chapter{Appendix}

The resulting artifacts of our research have been released as open-source software \cite{thesis-github}. Here we present a concise guide on how to run and use these artifacts.

Note that this guide is not meant to be universal, and instead shows the steps we performed to run the system in our specific setup.

\section{Running the Prototype}
For running the prototype we used the \textit{faas} flavour of \textit{OpenFaas}, which runs on top of \textit{Kubernetes}. If \textit{faasd}, the lighter version of \textit{OpenFaas}, is needed, then a slightly different setup would be necessary.


\subsection{Prerequisites}
The following applications and Command Line Interface programs are needed to setup the framework:
\begin{itemize}
    \item \textbf{npm}: the default package manager for the Node.js runtime environment;
    \\More info at this link: \href{https://docs.npmjs.com/downloading-and-installing-node-js-and-npm#using-a-node-version-manager-to-install-nodejs-and-npm}{docs.npmjs.com/downloading-and-installing-node-js-and-npm}

    \item \textbf{arkade}: a portable marketplace for downloading popular devops CLIs and installing helm charts;
    \\It can be installed with \inlinecode{curl -sLS https://get.arkade.dev | sudo sh};
    \\More info at this link: \href{https://github.com/alexellis/arkade}{github.com/alexellis/arkade}
    
    \item \textbf{faas-cli}: the Command Line Interface of \textit{OpenFaas};
    \\It can be installed with \inlinecode{arkade get faas-cli};
    
    \item \textbf{helm}: the \textit{Kubernetes} package manager;
    \\It can be installed with \inlinecode{arkade get helm};
    
    \item \textbf{minikube}: a local \textit{Kubernetes} engine;
    \\On our setup running macOS Big Sur with a x86-64 CPU it was installed with \inlinecode{brew install minikube};
    \\More info at this link: \href{https://minikube.sigs.k8s.io/docs/start/}{minikube.sigs.k8s.io/docs/start/}
    
    \item \textbf{deployer}: the CLI tool that we developed and that allows to deploy functions on the architecture;
    \\It can be installed by running the following command in the directory where the source code of the CLI is stored: \inlinecode{npm install -g};
\end{itemize}


\subsection{Kubernetes Setup}
To run \textit{Kubernetes} we used \textit{minikube}, a software which allows to easily create a Virtual Machine environment equipped with \textit{Kubernetes}. Note that in a production environment \textit{minikube} is not recommended, but in our emulation it was perfect to run multiple nodes that emulate the nodes in an edge network.

The following are the commands we used to start the Virtual Machines (note that as virtualization software connected to \textit{minikube} we used \textit{Parallels Desktop}):
\begin{lstlisting}[language=bash]
minikube delete --all  # Delete previous VMs

minikube config set driver parallels  # Set Parallels as virtualization software

minikube config set cpus 2  # Set 2 CPUs per VM

minikube config set memory 2048  # Set 2048MB of RAM per VM

minikube start --profile p1  # Start a new VM with name p1

minikube start --profile p2  # Start a new VM with name p2

minikube ip --profile p1  #  Get IP address of p1

minikube ip --profile p2  #  Get IP address of p2

kubectl config get-contexts  #  Print Kubernetes contexts (should show two Kubernetes machines, p1 and p2)

kubectl config use-context p1  #  Use p1 for the next commands

kubectl get po -A  # List all pods running on Kubernetes
\end{lstlisting}

At the end of these commands the results are, in this case, the creation of two empty VMs running \textit{Kubernetes}, without any external software installed on it.
Now the goal is to install the framework we developed on top of these \textit{Kubernetes} installations.

Three steps are still needed:
\begin{itemize}
    \item Install the \textit{faas} flavour of \textit{OpenFaas} on top of \textit{Kubernetes};
    \item Install \textit{Redis} on top of Kubernetes;
    \item Deploy on \textit{OpenFaas} the function we developed that allows locations to receive forwarded write actions.
\end{itemize}


\subsection{OpenFaas Setup}
On each \textit{Kubernetes} environment it is needed to install \textit{OpenFaas}. The installation can be performed in the following way:
\begin{lstlisting}[language=bash]
# Use p1 for the next commands (should be changed for every VM)
kubectl config use-context p1

# Apply OpenFaas configuration
kubectl apply -f https://raw.githubusercontent.com/openfaas/faas-netes/master/namespaces.yml

# Write OpenFaas password in a secret
kubectl -n openfaas create secret generic basic-auth --from-literal=basic-auth-user=admin --from-literal=basic-auth-password="customOpenFaasPassword"  

# Install OpenFaas
helm upgrade openfaas --install openfaas/openfaas --namespace openfaas --set functionNamespace=openfaas-fn --set basic_auth=true

# Install the cron addon
helm upgrade --install cron-connector openfaas/cron-connector --namespace openfaas

# Login to OpenFaas running on machine p1 that we have just installed
echo "customOpenFaasPassword" | faas-cli login -u admin --password-stdin --gateway http://$(minikube ip --profile p1):31112
\end{lstlisting}
At the end of this step we have the \textit{faas} flavour of \textit{OpenFaas} installed on every node.


\subsection{Redis Setup}
On each \textit{Kubernetes} environment it is needed to install \textit{Redis}. The installation can be performed in the following way:
\begin{lstlisting}[language=bash]
# Use p1 for the next commands (should be changed for every VM)
kubectl config use-context p1

# Install Redis
helm install my-openfaas-redis bitnami/redis --namespace openfaas-fn --set auth.password="customRedisPassword" --set master.persistence.enabled=false
\end{lstlisting}
At the end of this step we have a Redis instance installed on every node on top of \textit{Kubernetes}.
Now we can put all the IP addresses of the machines in the JSON of the infrastructure.
\\The IP addresses can be obtained, as seen, with \inlinecode{minikube ip --profile p1}.


\subsection{The Receiver Function}
After the infrastructure JSON file is ready we can deploy the function "edge-db-data-receiver" which allows locations to receive forwarded write actions.
\begin{lstlisting}[language=bash]
# Move in the directory where the "edge-db-data-receiver" function is stored
cd ./framework/functions-main/

# Build and publish the "edge-db-data-receiver" function on a Docker Registry
faas-cli publish --filter edge-db-data-receiver --platforms linux/arm/v7,linux/amd64

# Deploy the function on every level, except the lowest level
deployer deploy edge-db-data-receiver infrastructure.json --inEvery city
deployer deploy edge-db-data-receiver infrastructure.json --inEvery country
deployer deploy edge-db-data-receiver infrastructure.json --inEvery continent
\end{lstlisting}
At the end of this step the framework is ready to receive custom functions, that can be created by the developer as seen in Chapter \ref{ch:prototype}. 


\subsection{Deploying Custom Function}
To deploy new custom functions it is simply needed to perform the following commands:
\begin{lstlisting}[language=bash]
# Move in the directory where the "stack.yml" file is defined
cd ./framework/functions-main/

# Build and publish the function on a Docker Registry
faas-cli publish --filter my-function-name --platforms linux/arm/v7,linux/amd64

# Deploy the function
deployer deploy my-function-name infrastructure.json --inEvery district --inAreas italy paris --exceptIn milan
\end{lstlisting}


\section{Debugging the Prototype}

\subsection{Debugging Custom Functions}
The following commands can be used to print the logs of a function:
\begin{lstlisting}[language=bash]
# Select which node to debug
kubectl config use-context p1

# Print the logs of the function running on that node
kubectl logs -n openfaas-fn deploy/my-function-name -f
\end{lstlisting}
Or alternatively the \textit{faas-cli} can be used as follows:
\begin{lstlisting}[language=bash]
faas-cli logs my-function-name --gateway http://$(minikube ip --profile p1):31112
\end{lstlisting}


\subsection{Debugging the Framework}
To debug the framework itself or to understand why a function is having issues starting up, the following commands can be used:
\begin{lstlisting}[language=bash]
# Select which node to debug
kubectl config use-context p1

# List all pods and components running on Kubernetes
kubectl get po -A

# Print many info about a single component (in this example the gateway component)
kubectl describe -n openfaas deploy/gateway
kubectl logs -n openfaas deploy/gateway

# Print events happened in the openfaas namespace
kubectl get events -n openfaas --sort-by=.metadata.creationTimestamp
\end{lstlisting}


\section{Running the Simulation}
The Simulation is composed of many Phython files, one file for each scenario. To run a scenario it is simply possible to run the Python file with a Python IDE (e.g., PyCharm).

Note that a single configuration in some scenarios may simulate millions of machines and the simulation of such a huge number of machines requires an high utilization of RAM. For example the scenario included in \inlinecode{simulation\_read\_district\_ level\_clients\_ratio.py} can use up to 14 GB or RAM for the simulation.
If such usage of RAM becomes a problem, it is possible to make a modification to the trade-off between the speed of the execution of the simulation and the usage of RAM by modifying the following line:\\
\inlinecode{pool = multiprocessing.Pool(processes=4)}\\
and replacing the 4 with a lower number to use less RAM, while also using less parallelism for the simulation, resulting in a slower execution. In practice this number represents how many configurations of the scenario can be run in parallel on the given processes.


