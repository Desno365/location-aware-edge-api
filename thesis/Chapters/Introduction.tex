\chapter{Introduction}

\section{Context}

% This is basically an extension of the abstract. Here you provide context for the problem faced. Keep in mind that even if you now have gained expertise on it, most of the readers are no so inside the problem as you are. Start from the basics and explain clearly. You can also introduce here some hints about the methodology and your contribution. For this purpose, you may also decide to add more sections.

With the increasing number of connected devices and with \gls{IoT} implementation now becoming more widespread, in some cases cloud-centric architectures are starting to be ineffective. Devices are generating a lot of data at the end of the network and many applications are already being deployed at the edge to process the information.
Cisco Systems predicts that an estimated 29 billion devices will connect to the Internet by 2023 \cite{cisco2018-2023}.

Due to the volume, variety and velocity of data generated at the end of the network, the cloud cannot fully support applications that must meet compelling latency or bandwidth constraints: huge distances need to be covered by the communication, increasing the latency and making a large quantity of data pass through the network.
Indeed, the considerable increase in the amount of data produced at the end of the network was not accompanied by a comparable increase of available bandwidth from/to the cloud \cite{promise-of-edge-computing}.

% Furthermore, Cloud connection latencies are not adequate to host real-time tasks such as life-saving connected devices, augmented reality, or gaming [3].

% Some of the applications they run might require very short response times, some might involve private data, and some might produce huge quantities of data. Cloud computing can’t support these IoT applications. Edge computing, on the other hand, can do so and will promote many new IoT applications.

\section{Research Questions}
An increasing trend in edge computing has been found in the last years, however the industry lacks the presence of a stateful development abstraction that allows developers to easily exploit the power of the edge. The absence of this abstraction makes developers still prefer cloud-centric approach despite the related problems.

A non-technology and non-infrastructure dependant framework is needed in order to allow the development of applications with strict constraints of latency and bandwidth.

Therefore this work aims at answering the following research questions (RQ):
\begin{itemize}
    \item[RQ.1]\emph{Which use cases are predominantly affected by bandwidth and latency constraints? What are the common characteristics of these use cases?}
    
    \item[RQ.2]\emph{Which frameworks allowing computation on the edge are currently available in the industry? Can the available frameworks accomplish the use cases seen in RQ.1?}
    
    \item[RQ.3]\emph{Can a new approach accomplish the use cases seen in RQ.1? What are the drawbacks and benefits of the approach?}
\end{itemize}

We use the answers to these questions to propose an innovative framework that allows the developers to abstract away both the infrastructure and the location of the users.

\section{Research Methodology}
TODO

\section{Thesis Outline}
TODO



\iffalse
Here you explain the structure of the thesis.

\begin{example}
The thesis is structured in the following way:
\begin{itemize}
\item In \autoref{ch:preliminaries_and_sota}, we present ... .
\item In \autoref{ch:problem_formulation}, we formulate the problem we address in the thesis and ... .
\item In \autoref{ch:design}, we present our solution for ... .
\item In \autoref{ch:experiments}, we show experimental results of our proposed methods in different settings ... .
\item Finally, in \autoref{ch:conclusions}, we present our conclusions and possible future paths toward which our work could be extended.
\end{itemize}
\end{example}
\fi