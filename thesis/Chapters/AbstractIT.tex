\chapter*{Sommario}

La popolarità e la proliferazione di dispositivi intelligenti (e.g., smartphone, dispositivi indossabili, sensori Internet-of-Things) sta determinando una crescita senza precedenti della quantità di dati raccolti. Gli approcci attualmente più diffusi per gestire questa enorme quantità di dati si basano in genere su piattaforme cloud situate al centro dell'infrastruttura.

Con l'aumento del numero di dispositivi e della quantità di dati generati, tali approcci basati su un core centrale stanno diventando sempre più inefficienti poiché devono trasferire i dati avanti e indietro tra il core e i dispositivi. Inoltre, le latenze associate a tale trasferimento di dati sono influenzate dall'enorme distanza di viaggio necessaria per far comunicare il dispositivo con la piattaforma cloud centrale.

Per affrontare la situazione sono stati introdotti nuovi approcci sia nel mondo accademico che nell'industria, sfruttando la potenza dell'edge della rete per eseguire il calcolo più vicino alla fonte dei dati. Abbiamo notato una discrepanza tra gli approcci proposti nella ricerca e nell'industria: la ricerca presuppone spesso la possibilità di eseguire macchine virtuali o container di lunga durata sull'edge. Tuttavia, la maggior parte delle aziende di infrastruttura web non rispettano questa ipotesi a causa delle risorse limitate disponibili nell'edge.

In questa tesi studiamo lo stato dell'arte per le computazioni con stato e per l'elaborazione dei dati sull'edge, e dopo aver analizzato attentamente le problematiche e le esigenze dello scenario mostriamo i casi d'uso prevalentemente affetti da vincoli di larghezza di banda e latenza. Mostriamo quindi i framework attuali disponibili nel settore e notiamo come queste soluzioni non coprono i casi d'uso trovati. Quindi proponiamo un approccio serverless effettivamente applicabile dalle aziende di infrastrutture web, che tenga conto del problema della scarsità delle risorse pur consentendo computazioni stateful abbastanza potenti sull'edge. Mostriamo anche come abbiamo implementato questo nuovo approccio attraverso un prototipo funzionante, e infine esaminiamo i benefici che gli sviluppatori possono ottenere usando il nostro approccio. Dimostriamo come diversi casi d'uso possono trarre vantaggio da questo nuovo sistema attraverso la simulazione a eventi discreti, poiché l'esecuzione del nostro prototipo su un'emulazione di una rete edge globale era impossibile a causa dell'enorme quantità di risorse necessarie per emulare anche una piccola rete edge.

\textbf{Keywords:} Edge Computing, Serverless, FaaS, Stateful
