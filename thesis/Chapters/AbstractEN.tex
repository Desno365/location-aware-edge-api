\chapter*{Abstract}

The popularity and proliferation of smart devices (e.g., smartphones, wearable devices, Internet-of-Things sensors) is resulting in an unprecedented growth in the amount of collected data. The current most popular approaches to manage this huge amount of data typically rely on cloud platforms located at the core of the infrastructure.

As the number of devices and the amount of data they generate increases, such core-centric approaches are becoming increasingly inefficient as they need to transfer data back and forth between the core and the devices. Furthermore, the latencies associated with such data transfer are affected by the huge travel-distance needed to make the device communicate to the central cloud platform.

To deal with the aforementioned situation new approaches have been introduced in both academia and industry, exploiting the power of the edge of the network to perform the computation closer to the data source. We noticed a discrepancy between the approaches proposed in research and in industry: research frequently assumes the possibility of running virtual machines or long-running containers on the edge. However, most real-world web infrastructure companies do not comply with this assumption due to the limited resource available in the edge.

In this thesis we study the state of the art for stateful computations and data processing on the edge and after carefully analyzing the issues and the needs of the scenario we show the use cases predominantly affected by bandwidth and latency constraints. We then show the current frameworks available in the industry and notice how these solutions do not cover the use cases found. So we then propose a serverless approach effectively applicable by web infrastructure companies, that takes into consideration the problem of the scarcity of the resources, while still allowing quite powerful stateful computations on the edge. We also show how we implemented this new approach through a working prototype, and finally we investigate the gains developers may obtain by using our approach. We demonstrate how several use cases can benefit from this new system through discrete-event simulation, since running our prototype on an emulation of a global edge network was infeasible due to the sheer amount of resources needed to emulate even a small edge network.

\textbf{Keywords:} Edge Computing, Serverless, FaaS, Stateful



\iffalse
\chapter*{Abstract} 
Here goes the Abstract in English of your thesis followed by a list of keywords.
The Abstract is a concise summary of the content of the thesis (single page of text)
and a guide to the most important contributions included in your thesis.
The Abstract is the very last thing you write.
It should be a self-contained text and should be clear to someone who hasn't (yet) read the whole manuscript.
The Abstract should contain the answers to the main scientific questions that have been addressed in your thesis.
It needs to summarize the adopted motivations and the adopted methodological approach as well as the findings of your work and their relevance and impact.
The Abstract is the part appearing in the record of your thesis inside POLITesi,
the Digital Archive of PhD and Master Theses (Laurea Magistrale) of Politecnico di Milano.
The Abstract will be followed by a list of four to six keywords.
Keywords are a tool to help indexers and search engines to find relevant documents.
To be relevant and effective, keywords must be chosen carefully.
They should represent the content of your work and be specific to your field or sub-field.
Keywords may be a single word or two to four words. 
\\
\\
\textbf{Keywords:} here, the keywords, of your thesis % Keywords
\fi
