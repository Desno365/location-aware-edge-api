% A LaTeX template for MSc Thesis submissions to 
% Politecnico di Milano (PoliMi) - School of Industrial and Information Engineering
%
% S. Bonetti, A. Gruttadauria, G. Mescolini, A. Zingaro
% e-mail: template-tesi-ingind@polimi.it
%
% Last Revision: October 2021
%
% Copyright 2021 Politecnico di Milano, Italy. NC-BY

\documentclass{Configuration_Files/PoliMi3i_thesis}

%------------------------------------------------------------------------------
%	REQUIRED PACKAGES AND  CONFIGURATIONS
%------------------------------------------------------------------------------

% CONFIGURATIONS
\usepackage{parskip} % For paragraph layout
\usepackage{setspace} % For using single or double spacing
\usepackage{emptypage} % To insert empty pages
\usepackage{multicol} % To write in multiple columns (executive summary)
\setlength\columnsep{15pt} % Column separation in executive summary
\setlength\parindent{0pt} % Indentation
\raggedbottom  

% PACKAGES FOR TITLES
\usepackage{titlesec}
% \titlespacing{\section}{left spacing}{before spacing}{after spacing}
\titlespacing{\section}{0pt}{3.3ex}{2ex}
\titlespacing{\subsection}{0pt}{3.3ex}{1.65ex}
\titlespacing{\subsubsection}{0pt}{3.3ex}{1ex}
\usepackage{color}

% PACKAGES FOR LANGUAGE AND FONT
\usepackage[english]{babel} % The document is in English  
\usepackage[utf8]{inputenc} % UTF8 encoding
\usepackage[T1]{fontenc} % Font encoding
\usepackage[11pt]{moresize} % Big fonts

% PACKAGES FOR IMAGES
\usepackage{graphicx}
\usepackage{transparent} % Enables transparent images
\usepackage{eso-pic} % For the background picture on the title page
\usepackage{subfig} % Numbered and caption subfigures using \subfloat.
\usepackage{tikz} % A package for high-quality hand-made figures.
\usetikzlibrary{}
\graphicspath{{./Images/}} % Directory of the images
\usepackage{caption} % Coloured captions
\usepackage{xcolor} % Coloured captions
\usepackage{amsthm,thmtools,xcolor} % Coloured "Theorem"
\usepackage{float}

% STANDARD MATH PACKAGES
\usepackage{amsmath}
\usepackage{amsthm}
\usepackage{amssymb}
\usepackage{amsfonts}
\usepackage{bm}
\usepackage[overload]{empheq} % For braced-style systems of equations.
\usepackage{fix-cm} % To override original LaTeX restrictions on sizes

% PACKAGES FOR TABLES
\usepackage{tabularx}
\usepackage{longtable} % Tables that can span several pages
\usepackage{colortbl}

% PACKAGES FOR ALGORITHMS (PSEUDO-CODE)
\usepackage{algorithm}
\usepackage{algorithmic}

% PACKAGES FOR REFERENCES & BIBLIOGRAPHY
\usepackage[colorlinks=true,linkcolor=black,anchorcolor=black,citecolor=black,filecolor=black,menucolor=black,runcolor=black,urlcolor=black]{hyperref} % Adds clickable links at references
\usepackage{cleveref}
\usepackage[square, numbers, sort&compress]{natbib} % Square brackets, citing references with numbers, citations sorted by appearance in the text and compressed
\bibliographystyle{abbrvnat} % You may use a different style adapted to your field

% OTHER PACKAGES
\usepackage{pdfpages} % To include a pdf file
\usepackage{afterpage}
\usepackage{lipsum} % DUMMY PACKAGE
\usepackage{fancyhdr} % For the headers
\fancyhf{}

% Input of configuration file. Do not change config.tex file unless you really know what you are doing. 
% Set the geometric layout of the document
\usepackage{geometry}
\geometry{
  top=3cm,
  left = 2.0cm,
  right = 2.0cm,
  bottom=2cm,
  headheight= 2cm,
  headsep= 0cm,
}
\raggedbottom 

% Create color bluePoli (-> manuale grafica coordinata:  https://www.polimi.it/fileadmin/user_upload/il_Politecnico/grafica-coordinata/2015_05_11_46xy_manuale_grafica_coordinata.pdf)
\definecolor{bluePoli}{cmyk}{0.4,0.1,0,0.4}

% Custom theorem environments
\declaretheoremstyle[
  headfont=\color{bluePoli}\normalfont\bfseries,
  bodyfont=\color{black}\normalfont\itshape,
]{colored}

\captionsetup[figure]{labelfont={color=bluePoli}} % Set colour of the captions
\captionsetup[table]{labelfont={color=bluePoli}} % Set colour of the captions
\captionsetup[algorithm]{labelfont={color=bluePoli}} % Set colour of the captions

\theoremstyle{colored}
\newtheorem{theorem}{Theorem}[section]
\newtheorem{proposition}{Proposition}[section]

% Enhances the features of the standard "table" and "tabular" environments.
\newcommand\T{\rule{0pt}{2.6ex}}
\newcommand\B{\rule[-1.2ex]{0pt}{0pt}}

% Algorithm description
\newcounter{algsubstate}
\renewcommand{\thealgsubstate}{\alph{algsubstate}}
\newenvironment{algsubstates}{
    \setcounter{algsubstate}{0}%
    \renewcommand{\STATE}{%
    \stepcounter{algsubstate}%
    \Statex {\small\thealgsubstate:}\space}
    }{}
    
% Custom theorem environment
\newcolumntype{L}[1]{>{\raggedright\let\newline\\\arraybackslash\hspace{0pt}}m{#1}}
\newcolumntype{C}[1]{>{\centering\let\newline\\\arraybackslash\hspace{0pt}}m{#1}}
\newcolumntype{R}[1]{>{\raggedleft\let\newline\\\arraybackslash\hspace{0pt}}m{#1}}

% Custom itemize environment
\setlist[itemize,1]{label=$\bullet$}
\setlist[itemize,2]{label=$\circ$}
\setlist[itemize,3]{label=$-$}
\setlist{nosep}

% Set separation of columns 
\setlength{\columnsep}{30pt}

% Create command for background pic
\newcommand\BackgroundPic{% Adding background picture
	\put(230,358){
		\parbox[b][\paperheight]{\paperwidth}{%
			\vfill
			\centering
			\transparent{0.4}
			\includegraphics[width=0.5\paperwidth]{raggiera_polimi.eps}%
			\vfill
}}}

% Set indentation
\setlength\parindent{0pt}

% Custom title commands
\titleformat{\section}
{\color{bluePoli}\normalfont\Large\bfseries}
{\color{bluePoli}\thesection.}{1em}{}
\titlespacing*{\section}
{0pt}{2ex}{1ex}

\titleformat{\subsection}
{\color{bluePoli}\normalfont\large\bfseries}
{\color{bluePoli}\thesubsection.}{1em}{}
\titlespacing*{\subsection}
{0pt}{2ex}{1ex}

% Custom headers and footers
\pagestyle{fancy}
\fancyhf{}
      
\fancyfoot{}
\fancyfoot[C]{\thepage} % page
\renewcommand{\headrulewidth}{0mm} % headrule width
\renewcommand{\footrulewidth}{0mm} % footrule width

\makeatletter
\patchcmd{\headrule}{\hrule}{\color{black}\hrule}{}{} % headrule
\patchcmd{\footrule}{\hrule}{\color{black}\hrule}{}{} % footrule
\makeatother

% -> Create the header
\chead[C]{
\centering
\begin{tcolorbox}[arc=0pt, boxrule=0pt, colback=bluePoli!60, width=\textwidth, colupper=white]
    \textbf{Executive summary} \hfill \textbf{\author}  
\end{tcolorbox}
}

%----------------------------------------------------------------------------
%	NEW COMMANDS DEFINED
%----------------------------------------------------------------------------

% EXAMPLES OF NEW COMMANDS
\newcommand{\bea}{\begin{eqnarray}} % Shortcut for equation arrays
\newcommand{\eea}{\end{eqnarray}}
\newcommand{\e}[1]{\times 10^{#1}}  % Powers of 10 notation

%----------------------------------------------------------------------------
%	ADD YOUR PACKAGES (be careful of package interaction)
%----------------------------------------------------------------------------

\usepackage{array,tabularx}
\usepackage{longtable}
\usepackage{multirow}
\usepackage{svg}
\usepackage[utf8]{inputenc}
\usepackage{xcolor}

%----------------------------------------------------------------------------
%	ADD YOUR DEFINITIONS AND COMMANDS (be careful of existing commands)
%----------------------------------------------------------------------------

% Better inline code
\definecolor{light-gray}{gray}{0.95}
\newcommand{\inlinecode}[1]{\colorbox{light-gray}{\texttt{#1}}}

%----------------------------------------------------------------------------
%	BEGIN OF YOUR DOCUMENT
%----------------------------------------------------------------------------

\begin{document}

\fancypagestyle{plain}{%
\fancyhf{} % Clear all header and footer fields
\fancyhead[RO,RE]{\thepage} %RO=right odd, RE=right even
\renewcommand{\headrulewidth}{0pt}
\renewcommand{\footrulewidth}{0pt}}

%----------------------------------------------------------------------------
%	TITLE PAGE
%----------------------------------------------------------------------------

\pagestyle{empty} % No page numbers
\frontmatter % Use roman page numbering style (i, ii, iii, iv...) for the preamble pages

\puttitle{
	title=Location-Aware Stateful Serverless Computing on the Edge, % Title of the thesis
	name=Dennis Motta, % Author Name and Surname
	course=Computer Science and Engineering, % Study Programme (in Italian)
	ID  = 940064,  % Student ID number (numero di matricola)
	advisor= Prof. Alessandro Margara, % Supervisor name
	coadvisor={Gianpaolo Cugola}, % Co-Supervisor name, remove this line if there is none
	academicyear={2020-21},  % Academic Year
} % These info will be put into your Title page 

%----------------------------------------------------------------------------
%	PREAMBLE PAGES: ABSTRACT (inglese e italiano), EXECUTIVE SUMMARY
%----------------------------------------------------------------------------
\startpreamble
\setcounter{page}{1} % Set page counter to 1

% ABSTRACT IN ENGLISH
\chapter*{Abstract}

The popularity and proliferation of smart devices (e.g., smartphones, wearable devices, Internet-of-Things sensors) is resulting in an unprecedented growth in the amount of collected data. The current most popular approaches to manage this huge amount of data typically rely on cloud platforms located at the core of the infrastructure.

As the number of devices and the amount of data they generate increases, such core-centric approaches are becoming increasingly inefficient as they need to transfer data back and forth between the core and the devices. Furthermore, the latencies associated with such data transfer are affected by the huge travel-distance needed to make the device communicate to the central cloud platform.

To deal with the aforementioned situation new approaches have been introduced in both academia and industry, exploiting the power of the edge of the network to perform the computation closer to the data source. We noticed a discrepancy between the approaches proposed in research and in industry: research frequently assumes the possibility of running virtual machines or long-running containers on the edge. However, most real-world Web infrastructure companies do not comply with this assumption, due to the limited resource available in the edge.

In this thesis we study the state of the art for stateful computations and data processing on the edge and after carefully analyzing the issues and the needs of the scenario we show the use cases predominantly affected by bandwidth and latency constraints. We then show the current frameworks available in the industry and notice how these solutions do not cover the use cases found. So we then propose a serverless approach effectively applicable by web infrastructure companies, that takes into consideration the problem of the scarcity of the resources while still allowing quite powerful stateful computations on the edge. We also show how we implemented this new approach trough a working prototype, and finally we investigate the gains developers may obtain by using our approach. We demonstrate how several use cases can benefit from this new system though discrete-event simulation, since running our prototype on an emulation of a real global edge network was infeasible due to to sheer amount of resources needed to emulate even a small edge network.

\textbf{Keywords:} Edge Computing, Serverless, FaaS, Stateful



\iffalse
\chapter*{Abstract} 
Here goes the Abstract in English of your thesis followed by a list of keywords.
The Abstract is a concise summary of the content of the thesis (single page of text)
and a guide to the most important contributions included in your thesis.
The Abstract is the very last thing you write.
It should be a self-contained text and should be clear to someone who hasn't (yet) read the whole manuscript.
The Abstract should contain the answers to the main scientific questions that have been addressed in your thesis.
It needs to summarize the adopted motivations and the adopted methodological approach as well as the findings of your work and their relevance and impact.
The Abstract is the part appearing in the record of your thesis inside POLITesi,
the Digital Archive of PhD and Master Theses (Laurea Magistrale) of Politecnico di Milano.
The Abstract will be followed by a list of four to six keywords.
Keywords are a tool to help indexers and search engines to find relevant documents.
To be relevant and effective, keywords must be chosen carefully.
They should represent the content of your work and be specific to your field or sub-field.
Keywords may be a single word or two to four words. 
\\
\\
\textbf{Keywords:} here, the keywords, of your thesis % Keywords
\fi


% ABSTRACT IN ITALIAN
%\chapter*{Sommario}

La popolarità e la proliferazione di dispositivi intelligenti (e.g., smartphone, dispositivi indossabili, sensori Internet-of-Things) sta determinando una crescita senza precedenti della quantità di dati raccolti. Gli approcci attualmente più diffusi per gestire questa enorme quantità di dati si basano in genere su piattaforme cloud situate al centro dell'infrastruttura.

Con l'aumento del numero di dispositivi e della quantità di dati generati, tali approcci basati su un core centrale stanno diventando sempre più inefficienti poiché devono trasferire i dati avanti e indietro tra il core e i dispositivi. Inoltre, le latenze associate a tale trasferimento di dati sono influenzate dall'enorme distanza di viaggio necessaria per far comunicare il dispositivo con la piattaforma cloud centrale.

Per affrontare la situazione sono stati introdotti nuovi approcci sia nel mondo accademico che nell'industria, sfruttando la potenza dell'edge della rete per eseguire il calcolo più vicino alla fonte dei dati. Abbiamo notato una discrepanza tra gli approcci proposti nella ricerca e nell'industria: la ricerca presuppone spesso la possibilità di eseguire macchine virtuali o container di lunga durata sull'edge. Tuttavia, la maggior parte delle aziende di infrastruttura web non rispettano questa ipotesi a causa delle risorse limitate disponibili nell'edge.

In questa tesi studiamo lo stato dell'arte per le computazioni con stato e per l'elaborazione dei dati sull'edge, e dopo aver analizzato attentamente le problematiche e le esigenze dello scenario mostriamo i casi d'uso prevalentemente affetti da vincoli di larghezza di banda e latenza. Mostriamo quindi i framework attuali disponibili nel settore e notiamo come queste soluzioni non coprono i casi d'uso trovati. Quindi proponiamo un approccio serverless effettivamente applicabile dalle aziende di infrastrutture web, che tenga conto del problema della scarsità delle risorse pur consentendo computazioni stateful abbastanza potenti sull'edge. Mostriamo anche come abbiamo implementato questo nuovo approccio attraverso un prototipo funzionante, e infine esaminiamo i benefici che gli sviluppatori possono ottenere usando il nostro approccio. Dimostriamo come diversi casi d'uso possono trarre vantaggio da questo nuovo sistema attraverso la simulazione a eventi discreti, poiché l'esecuzione del nostro prototipo su un'emulazione di una rete edge globale era impossibile a causa dell'enorme quantità di risorse necessarie per emulare anche una piccola rete edge.

\textbf{Keywords:} Edge Computing, Serverless, FaaS, Stateful


% ACKNOWLEDGEMENTS
\chapter*{Acknowledgements}
Special thanks to Prof.~Alessandro Margara and Prof.~Gianpaolo Cugola for the professionalism and dedication with which they guided this project.

We also would like to thank Giampietro Fabrizio Bonaccorsi e Leonardo Barilani who are currently working on continuing this project and who provided valuable ideas for future works and improvements.

Last but not least, we want to remind that this thesis has also been possible thanks to the resources provided by Politecnico di Milano and to the knowledge acquired at this outstanding University.


%----------------------------------------------------------------------------
%	LIST OF CONTENTS/FIGURES/TABLES/SYMBOLS
%----------------------------------------------------------------------------

% TABLE OF CONTENTS
\thispagestyle{empty}
\tableofcontents % Table of contents 
\thispagestyle{empty}
\cleardoublepage

%-------------------------------------------------------------------------
%	THESIS MAIN TEXT
%-------------------------------------------------------------------------
% In the main text of your thesis you can write the chapters in two different ways:
%
%(1) As presented in this template you can write:
%    \chapter{Title of the chapter}
%    *body of the chapter*
%
%(2) You can write your chapter in a separated .tex file and then include it in the main file with the following command:
%    \chapter{Title of the chapter}
%    \input{chapter_file.tex}
%
% Especially for long thesis, we recommend you the second option.

\addtocontents{toc}{\vspace{2em}} % Add a gap in the Contents, for aesthetics
\mainmatter % Begin numeric (1,2,3...) page numbering

% --------------------------------------------------------------------------
% NUMBERED CHAPTERS % Regular chapters following
% --------------------------------------------------------------------------
\chapter{Introduction}

\section{Context}

% This is basically an extension of the abstract. Here you provide context for the problem faced. Keep in mind that even if you now have gained expertise on it, most of the readers are no so inside the problem as you are. Start from the basics and explain clearly. You can also introduce here some hints about the methodology and your contribution. For this purpose, you may also decide to add more sections.

With the increasing number of connected devices and with Internet-of-Thing (IoT) implementation now becoming more widespread, in some cases cloud-centric architectures are starting to be ineffective. Devices are generating a lot of data at the end of the network and many applications are already being deployed at the edge to process the information.
Cisco Systems predicts that an estimated 29 billion devices will connect to the Internet by 2023 \cite{cisco2018-2023}.

Due to the volume, variety and velocity of data generated at the end of the network, the cloud cannot fully support applications that must meet compelling latency or bandwidth constraints: huge distances need to be covered by the communication, increasing the latency and making a large quantity of data pass through the network.
Indeed, the considerable increase in the amount of data produced at the end of the network was not accompanied by a comparable increase of available bandwidth from/to the cloud \cite{promise-of-edge-computing}.

% Furthermore, Cloud connection latencies are not adequate to host real-time tasks such as life-saving connected devices, augmented reality, or gaming [3].

% Some of the applications they run might require very short response times, some might involve private data, and some might produce huge quantities of data. Cloud computing can’t support these IoT applications. Edge computing, on the other hand, can do so and will promote many new IoT applications.

\section{Research Questions}
An increasing trend in edge computing has been found in the last years, however the industry lacks the presence of a stateful development abstraction that allows developers to easily exploit the power of the edge. The absence of this abstraction makes developers still prefer cloud-centric approach despite the related problems.

A non-technology and non-infrastructure dependant framework is needed in order to allow the development of applications with strict constraints of latency and bandwidth.

Therefore this work aims at answering the following research questions (RQ):
\begin{itemize}
    \item[RQ.1]\emph{Which use cases are predominantly affected by bandwidth and latency constraints? What are the common characteristics of these use cases?}
    
    \item[RQ.2]\emph{Which frameworks allowing computation on the edge are currently available in the industry? Can the available frameworks accomplish the use cases seen in RQ.1?}
    
    \item[RQ.3]\emph{Can a new approach accomplish the use cases seen in RQ.1? What are the drawbacks and benefits of the approach?}
\end{itemize}

We use the answers to these questions to propose an innovative framework that allows the developers to abstract away both the infrastructure and the location of the users.

\section{Research Methodology}
TODO

\section{Thesis Outline}
TODO



\iffalse
Here you explain the structure of the thesis.

\begin{example}
The thesis is structured in the following way:
\begin{itemize}
\item In \autoref{ch:preliminaries_and_sota}, we present ... .
\item In \autoref{ch:problem_formulation}, we formulate the problem we address in the thesis and ... .
\item In \autoref{ch:design}, we present our solution for ... .
\item In \autoref{ch:experiments}, we show experimental results of our proposed methods in different settings ... .
\item Finally, in \autoref{ch:conclusions}, we present our conclusions and possible future paths toward which our work could be extended.
\end{itemize}
\end{example}
\fi
%\chapter{Preliminaries and State of the Art}
\label{ch:preliminaries_and_sota}

\section{Preliminary notions}
\label{sec:preliminaries}

\begin{table}[!ht]
\centering
\begin{tabular}{c l} \hline
\textbf{Notation}&\textbf{Description} \\ \hline
$G$&Graph\\
$V$&set of nodes of $G$\\
$E$&set of edges of $G$\\
$W$&set of weights corresponding to each edge in $E$\\
$w_{u,v}$&weight of edge $(u,v)$\\
$n$&$|V|$, number of nodes\\
$m$&$|E|$, number of edges\\
\hline
\end{tabular}
\caption{Graph notation.}
\label{tab:notation}
\end{table}

``In this section, we introduce the preliminary notions at the base of our study. We start by briefly introducing the problem, and then we provide the necessary concepts and the notation used."

You may insert a subsection for each of the most relevant features of your problem. You can add some reference if needed, but just to explain the problem. The references with the solutions of the problem should be put in the next section.

You can keep a notation table for the notation used in this chapter as \autoref{tab:notation}. Everything inside the notation table must be written at least once inside this chapter. You can put an extended notation for the whole thesis in the appendix.

It is likely that you have to present definitions, theorems or propositions. We suggests to use the environments provided by the template. You can find the guide in the LaTeX suggestions chapter.

\section{State of the Art}
\label{sec:sota}

In this section, we survey the most relevant works related to the argument of your thesis. If you face a problem that has more than one macro-topic, you may choose to add a subsection for each of these topics (better no more than 2-3), like \emph{Related works on Topic 1}, etc.

List the works in chronological order and cite only the most important and pertinent ones, avoid 100 citations for a master thesis.
%\chapter{Tips}
\label{ch:chapter_one}%
% The \label{...}% enables to remove the small indentation that is generated, always leave the % symbol.

In this chapter additional useful information are reported.

\section{Sections and subsections}
\label{sec:section_name}
Chapters are typically subdivided into sections and subsections, and, optionally,
subsubsections, paragraphs and subparagraphs.
All can have a title, but only sections and subsections are numbered.
A new section is created by the command
\begin{verbatim}
\section{Title of the section}
\end{verbatim}
The numbering can be turned off by using \verb|\section*{}|.
\\
A new subsection is created by the command
\begin{verbatim}
\subsection{Title of the subsection}
\end{verbatim}
and, similarly, the numbering can be turned off by adding an asterisk as follows 
\begin{verbatim}
\subsection*{}
\end{verbatim}

\section{Equations}
\label{sec:eqs}
This section gives some examples of writing mathematical equations in your thesis.

Maxwell's equations read:
\begin{subequations}
    \label{eq:maxwell}
    \begin{align}[left=\empheqlbrace]
    \nabla\cdot \bm{D} & = \rho, \label{eq:maxwell1} \\
    \nabla \times \bm{E} +  \frac{\partial \bm{B}}{\partial t} & = \bm{0}, \label{eq:maxwell2} \\
    \nabla\cdot \bm{B} & = 0, \label{eq:maxwell3} \\
    \nabla \times \bm{H} - \frac{\partial \bm{D}}{\partial t} &= \bm{J}. \label{eq:maxwell4}
    \end{align}
\end{subequations}

Equation~\eqref{eq:maxwell} is automatically labeled by \texttt{cleveref},
as well as Equation~\eqref{eq:maxwell1} and Equation~\eqref{eq:maxwell3}.
Thanks to the \verb|cleveref| package, there is no need to use \verb|\eqref|.
Remember that Equations have to be numbered only if they are referenced in the text.

Equations~\eqref{eq:maxwell_multilabels1}, \eqref{eq:maxwell_multilabels2}, \eqref{eq:maxwell_multilabels3}, and \eqref{eq:maxwell_multilabels4} show again Maxwell's equations without brace:
\begin{align}
    \nabla\cdot \bm{D} & = \rho, \label{eq:maxwell_multilabels1} \\
    \nabla \times \bm{E} +  \frac{\partial \bm{B}}{\partial t} &= \bm{0}, \label{eq:maxwell_multilabels2} \\
    \nabla\cdot \bm{B} & = 0, \label{eq:maxwell_multilabels3} \\
    \nabla \times \bm{H} - \frac{\partial \bm{D}}{\partial t} &= \bm{J} \label{eq:maxwell_multilabels4}.
\end{align}

Equation~\eqref{eq:maxwell_singlelabel} is the same as before,
but with just one label:
\begin{equation}
    \label{eq:maxwell_singlelabel}
    \left\{
    \begin{aligned}
    \nabla\cdot \bm{D} & = \rho, \\
    \nabla \times \bm{E} +  \frac{\partial \bm{B}}{\partial t} &= \bm{0},\\
    \nabla\cdot \bm{B} & = 0, \\
    \nabla \times \bm{H} - \frac{\partial \bm{D}}{\partial t} &= \bm{J}.
    \end{aligned}
    \right.
\end{equation}

\section{Figures, Tables and Algorithms}
Figures, Tables and Algorithms have to contain a Caption that describe their content, and have to be properly reffered in the text.

\subsection{Figures}
\label{subsec:figures}

For including pictures in your text you can use \texttt{TikZ} for high-quality hand-made figures,
or just include them as usual with the command
\begin{verbatim}
\includegraphics[options]{filename.xxx}
\end{verbatim}
Here xxx is the correct format, e.g. \verb|.png|, \verb|.jpg|, \verb|.eps|, \dots.

\begin{figure}[H]
    \centering
    \includegraphics[width=0.3\textwidth]{logo_polimi_scritta.eps}
    \caption{Caption of the Figure to appear in the List of Figures.}
    \label{fig:quadtree}
\end{figure}

Thanks to the \texttt{\textbackslash subfloat} command, a single figure, such as Figure~\ref{fig:quadtree},
can contain multiple sub-figures with their own caption and label, e.g. \color{black} Figure~\ref{fig:polimi_logo1} and Figure~\ref{fig:polimi_logo2}. 

\begin{figure}[H]
    \centering
    \subfloat[One PoliMi logo.\label{fig:polimi_logo1}]{
        \includegraphics[scale=0.5]{Images/logo_polimi_scritta.eps}
    }
    \quad
    \subfloat[Another one PoliMi logo.\label{fig:polimi_logo2}]{
        \includegraphics[scale=0.5]{Images/logo_polimi_scritta2.eps}
    }
    \caption[Shorter caption]{This is a very long caption you don't want to appear in the List of Figures.}
    \label{fig:quadtree2}
\end{figure}


\subsection{Tables}
\label{subsec:tables}

Within the environments \texttt{table} and  \texttt{tabular} you can create very fancy tables as the one shown in Table~\ref{table:example}.
\begin{table}[H]
    \caption*{\textbf{Title of Table (optional)}}
    \centering 
    \begin{tabular}{|p{3em} c c c |}
    \hline
    \rowcolor{bluepoli!40} % comment this line to remove the color
     & \textbf{column 1} & \textbf{column 2} & \textbf{column 3} \T\B \\
    \hline \hline
    \textbf{row 1} & 1 & 2 & 3 \T\B \\
    \textbf{row 2} & $\alpha$ & $\beta$ & $\gamma$ \T\B\\
    \textbf{row 3} & alpha & beta & gamma \B\\
    \hline
    \end{tabular}
    \\[10pt]
    \caption{Caption of the Table to appear in the List of Tables.}
    \label{table:example}
\end{table}

You can also consider to highlight selected columns or rows in order to make tables more readable.
Moreover, with the use of \texttt{table*} and the option \texttt{bp} it is possible to align them at the bottom of the page. One example is presented in Table~\ref{table:exampleC}. 

\begin{table}[H]
\centering 
    \begin{tabular}{|p{3em} | c | c | c | c | c | c|}
    \hline
%    \rowcolor{bluepoli!40}
     & \textbf{column1} & \textbf{column2} & \textbf{column3} & \textbf{column4} & \textbf{column5} & \textbf{column6} \T\B \\
    \hline \hline
    \textbf{row1} & 1 & 2 & 3 & 4 & 5 & 6 \T\B\\
    \textbf{row2} & a & b & c & d & e & f \T\B\\
    \textbf{row3} & $\alpha$ & $\beta$ & $\gamma$ & $\delta$ & $\phi$ & $\omega$ \T\B\\
    \textbf{row4} & alpha & beta & gamma & delta & phi & omega \B\\
    \hline
    \end{tabular}
    \\[10pt]
    \caption{Highlighting the columns}
    \label{table:exampleC}
\end{table}

\begin{table}[H]
\centering 
    \begin{tabular}{|p{3em} c c c c c c|}
    \hline
%    \rowcolor{bluepoli!40}
     & \textbf{column1} & \textbf{column2} & \textbf{column3} & \textbf{column4} & \textbf{column5} & \textbf{column6} \T\B \\
    \hline \hline
    \textbf{row1} & 1 & 2 & 3 & 4 & 5 & 6 \T\B\\
    \hline
    \textbf{row2} & a & b & c & d & e & f \T\B\\
    \hline
    \textbf{row3} & $\alpha$ & $\beta$ & $\gamma$ & $\delta$ & $\phi$ & $\omega$ \T\B\\
    \hline
    \textbf{row4} & alpha & beta & gamma & delta & phi & omega \B\\
    \hline
    \end{tabular}
    \\[10pt]
    \caption{Highlighting the rows}
    \label{table:exampleR}
\end{table}

\subsection{Algorithms}
\label{subsec:algorithms}

Pseudo-algorithms can be written in \LaTeX{} with the \texttt{algorithm} and \texttt{algorithmic} packages.
An example is shown in Algorithm~\ref{alg:var}.
\begin{algorithm}[H]
    \label{alg:example}
    \caption{Name of the Algorithm}
    \label{alg:var}
    \label{protocol1}
    \begin{algorithmic}[1]
    \STATE Initial instructions
    \FOR{$for-condition$}
    \STATE{Some instructions}
    \IF{$if-condition$}
    \STATE{Some other instructions}
    \ENDIF
    \ENDFOR
    \WHILE{$while-condition$}
    \STATE{Some further instructions}
    \ENDWHILE
    \STATE Final instructions
    \end{algorithmic}
\end{algorithm} 

\vspace{5mm}

\section{Theorems, propositions and lists}

\subsection{Theorems}
Theorems have to be formatted as:
\begin{theorem}
\label{a_theorem}
Write here your theorem. 
\end{theorem}
\textit{Proof.} If useful you can report here the proof.

\subsection{Propositions}
Propositions have to be formatted as:
\begin{proposition}
Write here your proposition.
\end{proposition}

\subsection{Lists}
How to  insert itemized lists:
\begin{itemize}
    \item first item;
    \item second item.
\end{itemize}
How to insert numbered lists:
\begin{enumerate}
    \item first item;
    \item second item.
\end{enumerate}

\section{Use of copyrighted material}

Each student is responsible for obtaining copyright permissions, if necessary, to include published material in the thesis.
This applies typically to third-party material published by someone else.

\section{Plagiarism}

You have to be sure to respect the rules on Copyright and avoid an involuntary plagiarism.
It is allowed to take other persons' ideas only if the author and his original work are clearly mentioned.
As stated in the Code of Ethics and Conduct, Politecnico di Milano \textit{promotes the integrity of research,
condemns manipulation and the infringement of intellectual property}, and gives opportunity to all those
who carry out research activities to have an adequate training on ethical conduct and integrity while doing research.
To be sure to respect the copyright rules, read the guides on Copyright legislation and citation styles available
at:
\begin{verbatim}
https://www.biblio.polimi.it/en/tools/courses-and-tutorials
\end{verbatim}
You can also attend the courses which are periodically organized on "Bibliographic citations and bibliography management".

\section{Bibliography and citations}
Your thesis must contain a suitable Bibliography which lists all the sources consulted on developing the work.
The list of references is placed at the end of the manuscript after the chapter containing the conclusions.
We suggest to use the BibTeX package and save the bibliographic references  in the file \verb|Thesis_bibliography.bib|.
This is indeed a database containing all the information about the references. To cite in your manuscript, use the \verb|\cite{}| command as follows:
\\
\textit{Here is how you cite bibliography entries: \cite{knuth74}, or multiple ones at once: \cite{knuth92,lamport94}}.
\\
The bibliography and list of references are generated automatically by running BibTeX \cite{bibtex}.
%\chapter{Problem formulation}
\label{ch:problem_formulation}

This chapter is dedicated to the formal presentation of the problem with the technical details. Here your should put also the figure of merit you use to compare your solution with the ones of the related works you presented.

\begin{example}
In this thesis, we address the problem of ... . [Description of the problem and classical approaches]. The figure of merit we use to compare the solutions is ... .
\end{example}

%\chapter{Design of the Solution}
\label{ch:design}

Let’s imagine a global infrastructure with many edge locations. Some web infrastructure companies already provide a network with many locations, like the one of Cloudflare with more than 250 locations worldwide located in more than 100 different countries \cite{cloudflare-network}, or the network of Amazon Web Services (AWS) with more than 265 edge locations \cite{aws-network}.
These networks are only the beginning of the development of edge networks: AWS is currently introducing AWS Wavelength \cite{aws-wavelength} a new service in partnership with popular telecommunication providers (i.e., Verizon, Vodafone) to scale across global 5G networks.


\section{Managing the network}
By having such a vast and heterogeneous network the first step in our approach is to abstract away the difficult management of the deployment to the many edge locations.
In a traditional cloud setup the developer specifies individually on which data center to deploy, this cannot be done efficiently for a vast edge network because the developer would have to specify hundreds of specific deployments.
The developer could also want to use only more powerful data centers and not the limited cloudlets at the border of the network.

\begin{wrapfigure}[11]{r}{5.5cm} % [11] is how many lines should be wrapped
\caption{An example of the hierarchy.}
\label{fig:hierarchy}
\includegraphics[width=5.5cm]{Figures/Solution/hierarchy.png}
\end{wrapfigure} 

To allow flexibility in the deployments and, as we will see later, to allow geographical aggregations we can organize the various machines running in the data centers and cloudlets in a hierarchy with multiple levels.
In Figure \ref{fig:hierarchy} we reported an example: we first have a division by continent (or large regions), then by country, territory, city and district.
Note that each element in the hierarchy should not necessarily be a different data center: a big data center in Milan can be both a receiver for "city" and "country" deployments/aggregations.
\vspace{0.5cm}


\subsection{Specifying locations}
The job of specifying the available locations should be responsibility of the web infrastructure company, but still the developer may want to customize the arrangement or may want to use their personal infrastructure. 

So we must provide a way to specify the hierarchy, we chose to implement the API in the following way:
\begin{itemize}
    \item Levels: the list of levels characterized by their identifiers (e.g., "continent", "country", "territory", "city", "district");
    \item Hierarchy: a way to specify the hierarchy, starting from the uppermost level and going down to the lowest level. Each level can contain multiple locations, and each location will aggregate data of the relative area;
    \item Location: each location must be associated to an entry point that defines the actual cloudlet or data center to be used (so for the location there must be defined gateway, port and password).
\end{itemize}

TODO show constraints


\subsection{Specifying deployments}
Now that we have the hierarchy specified we can use this hierarchy to make a powerful deployment API. The developer should be able to deploy on a specific level of the hierarchy only in a certain area and to exclude a specific subsection from this area.

To allow such deployment we established the following concepts:
\begin{itemize}
    \item \inlinecode{inEvery}: a string representing the level on which to deploy the function.
    \item \inlinecode{inAreas}: a list of string of areas, specifying in which areas to deploy. If unspecified we can assume the developer wants it deployed everywhere. 
     \item \inlinecode{exceptIn}: a list of string of areas, that are an exception to what previously defined before. In the locations contained in these areas the developer does not want to perform the deployment.
\end{itemize}

\begin{example}
Deploy on every city in Europe and Asia, excluding the cities in Italy and excluding the city of Tokyo.
Becomes:
\begin{itemize}
    \item \inlinecode{inEvery}: "city"
    \item \inlinecode{inAreas}: ["europe", "asia"]
     \item \inlinecode{exceptIn}: ["italy", "tokyo"]
\end{itemize}
\end{example}

\begin{example}
Deploy on every district in Europe and India, excluding the districts in France and excluding the districts in Milan.
Becomes:
\begin{itemize}
    \item \inlinecode{inEvery}: "district"
    \item \inlinecode{inAreas}: ["europe", "india"]
     \item \inlinecode{exceptIn}: ["france", "milan"]
\end{itemize}
\end{example}

In the examples we saw that the developer should also be allowed to mix the levels of the areas specified, using different levels of the hierarchy inside the \inlinecode{inAreas} and \inlinecode{exceptIn} lists.

TODO show constraints


\section{Managing limited resources}
Usually edge locations have limited resources compared to central data centers, so web infrastructure companies have to work around the limitations in order to provide a reliable service.
Due to these limitations, Function-as-a-Service (FaaS) is the current de facto standard for companies that provide computing resources at the edge to the public. Examples of services are AWS Lambda@Edge (an evolution of the famous AWS Lambda service on the cloud) \cite{aws-lambda-at-edge} or Cloudflare Workers \cite{cloudflare-workers}. It can be easily understood that providing Infrastructure-as-a-Service (IaaS) or Platform-as-a-Service (PaaS) to the public on the limited resources available at the edge is clearly less efficient for companies.

Therefore we decided to use in our framework the FaaS paradigm as a way to allow the users of the framework (the developers) to perform computation on the edge. The developer, should also be able to specify the RAM memory allocated for the function. The default allocated memory can be a low value, but if there is a more complex function requiring additional memory usage the developer can change the allocated memory (the web infrastructure company can then charge more based on the memory requested).

But resources are not only limited in the sense of computation, but also the storage resources are limited on the edge. To take into account the aforementioned issue we decided to use a key-value database for the stateful part of our approach. A key-value database allows us to perform extremely efficient (but simple) queries, and is perfect for the limited resources available on the edge. To avoid to accumulate data we also introduced in our framework a Time-To-Live (TTL) which the developer must specify. For example with a TTL of 5 days, after making a write to the database the data, if not updated, can be deleted after 5 days.


\subsection{Writing data}
Our goal is to provide the developers an easy way to create geographical aggregations of data. If for the data it does not make sense to create geographical aggregations then it would make more sense to use a core-centric approach to manage this data. In our approach by having the data geographically distributed it means that those data correspond to information coming from the respective geographic area.

Furthermore in Chapter \ref{ch:problem_formulation} we outlined use cases which are static in the sense of location, or for which having discontinuity in the data is not a problem. Therefore we decided to not introduce any session consistency to avoid the cost of managing such sessions: heavy communications between the locations. This means that if we are performing a "city" aggregation and a client that is sending data is currently travelling and changes its position to a new "city" area, then its old data will remain the in the previous "city" area and not transferred to the new one.

To manage such conditions our approach should have the following properties:
\begin{itemize}
    \item Write Action: the action to perform for writing data (e.g., set, add to list, ...)
    \item Key: The key that will be associated to the value as in a standard key-value database.
     \item Data: the data to be associated to the key (so it can be then obtained from the key) or the data to be added to the list specified by the key.
     \item Referring Area Level: the higher level on which to aggregate the data.
     \item Should Save In Intermediate Levels: a true or false property that if set to true saves the data only in the level specified by the Referring Area Level, otherwise the data is saved on the receiving level and on all the other upper levels, up until the level specified by Referring Area Level.
     \item Time To Live: limits the lifespan of the data.
\end{itemize}

To provide some examples, in the code that will be deployed on the "city" level the developer can make the following write calls:

\begin{example}
\begin{itemize}
    \item Write Action: set
    \item Key: mykey1
    \item Data: data1
    \item Referring Area Level: continent
    \item Should Save In Intermediate Levels: false
    \item Time To Live: 10 days
\end{itemize}
The framework will save the data only at the continent level (so if the code is executed in the "milan" location, the data will be saved in the "europe" location). The framework will create (or update) key "mykey1" with the value "data1" and will set a lifespan of the data to 10 days.
\end{example}

\begin{example}
\begin{itemize}
    \item Write Action: set
    \item Key: mykey2
    \item Data: data2
    \item Referring Area Level: continent
    \item Should Save In Intermediate Levels: true
    \item Time To Live: 30 days
\end{itemize}
The framework will save the data at the continent, country, territory and city levels (so if the code is executed in the "milan" location, the data will be saved in the "it-north", "italy" and "europe" locations). In each of these locations the framework will create (or update) key "mykey2" with the value "data2" and will set a lifespan of the data to 30 days.
\end{example}

\begin{example}
\begin{itemize}
    \item Write Action: add to list
    \item Key: mylist1
    \item Data: data1
    \item Referring Area Level: continent
    \item Should Save In Intermediate Levels: false
    \item Time To Live: 1 day
\end{itemize}
The framework will save the data only at the continent level (so if the code is executed in the "milan" location, the data will be saved in the "europe" location). The framework will add to the list associated to the key "mylist1" the value "data1" and will set a lifespan to the single element of the list 1 day.
\end{example}


\subsection{Reading data}
As we have seen the developer has to think carefully how to handle the writing of the data with the objective to partition geographically the data. But after data have been thoughtfully partitioned then the reading of the data is instantaneous.
When running the code in a certain location, by specifying only the reading action and key the framework obtains the data present in the location and associated to that key.

So only the following two properties are use for making reads:
\begin{itemize}
    \item Read Action: the action to perform for reading data, "get" to obtain the single value from the key (saved with "set"), "get list" to obtain all the values saved with "add to list".
    \item Key: The key of the value or the list to be read.
\end{itemize}

To provide some examples, in the code that will be deployed on the "continent" level, for reading the data saved in that "continent" location (e.g., "europe" location) the developer can make the following read calls:
\begin{example}
\begin{itemize}
    \item Read Action: get
    \item Key: mykey1
\end{itemize}
The framework will obtain the value associated to the key "mykey1" if present and if not expired due to the TTL.
\end{example}

\begin{example}
\begin{itemize}
    \item Read Action: get list
    \item Key: mylist1
\end{itemize}
The framework will obtain all the not expired elements contained in the list associated to the key "mylist1".
\end{example}


\section{Applying to use cases}
TODO show how these APIs can be used in a few use cases











\iffalse
``This approach is similar to the one proposed in a recent work by Nuara et al.~\cite{nuara2020driving}, in which ..."
\fi

\chapter{Experimental evaluation}
\label{ch:experiments}

This chapter shows the results of the experiments carried out to assess your contribution, usually compared to state of the art solutions. In the proposed structure, we separate experiments setup from experiment results, in which we list the experiments. This is especially convenient in case the setting is the same or very similar for all the experiments. Should this not be the case, you may consider instead to structure each experiment is a section/subsection and embed both setting and result inside it.

\begin{example}
In this chapter, we present experimental results on the algorithms proposed and we compare them with state of the art methods.
\end{example}

\section{Experimental setting}

First, describe the experimental setting. The setting usually contains:
\begin{itemize}
\item the dataset(s) used for the experiments;
\item the baselines, namely the other solutions with which you are comparing yours.
\end{itemize}

\begin{example}

\begin{table}[H]
\centering
\renewcommand{\arraystretch}{1.2}
\begin{tabular}{|l|c|c|c|c|c|}\hline
\textbf{Name}&$n$&$m$\\ \hline\hline
Email&1005&25571\\ \hline
\end{tabular}
\caption{Dataset used for the experiment.}
\label{tb:elf_datasets}
\end{table}

\paragraph{Datasets} \autoref{tb:elf_datasets} shows the characteristics of the real-world dataset used for the experiment. The dataset, provided by SNAP \cite{snapnets}, \emph{email-Eu-core}, has been generated using email data from a large European research institution. A directed edge $(u, v)$ means that person $u$ sent an e-mail to person $v$.

As widely done in literature, we assigned the ground-truth influence probabilities according to the weighted cascade model, that is, $p_{u,v}=\frac{1}{|In(v)|}$, for each edge $(u,v) \in E$.

\paragraph{Algorithms}

For the learning process, we use the Thompson Sampling (TS) as principal exploration strategy (\autoref{alg:cts}). For the sake of completeness, we show also results with a Pure Exploitation (PE) approach, in which the oracle is fed with the mean estimates of the influence probabilities at each round.

\end{example}

\section{Results}

In this section, we show and comment the results obtained from the experiments. Remember to comment every result and every figure you decide to insert in the thesis.

\begin{example}
\paragraph{Experiment 1}

\begin{figure}[H]
  \centering
  \includegraphics[width=.65\textwidth]{experiments/regret.pdf}
\caption{Cumulative regret in Email-In4 with 95\% confidence interval.}
\label{fig:exp_email}
\end{figure}

In this experiment, we test the algorithms on Email over a time horizon of $T=100$ rounds. The objective is to show the performances of the algorithms. The results have been averaged over 30 runs. \autoref{fig:exp_email} shows the cumulative regret, with a 95\% confidence interval.

As shown in the plot, our algorithm performs better with both the exploration strategies. However, the gain on TIM is more evident with the PE strategy, specially in the first rounds.
\end{example}
%\chapter{Conclusions and Future Works}
\label{ch:conclusions}

In this chapter, you present the conclusions of your thesis and a couple of possible future works to extend your results. First of all, you should briefly repeat the problem you addressed in the thesis. Then, you report your achievements and how they improve the state of the art.

\section{Conclusions}
\begin{example}
In this thesis, we analyzed the problem of ... . We proposed a new approach that ... . We tested this method on ... . Reported results show that our proposal outperforms the state of the art method.
\end{example}

\section{Future works}
\begin{example}
There are several appealing paths for future works. A possible extension could be to ... .
\end{example}


%-------------------------------------------------------------------------
%	BIBLIOGRAPHY
%-------------------------------------------------------------------------

\addtocontents{toc}{\vspace{2em}} % Add a gap in the Contents, for aesthetics
\bibliography{Thesis_bibliography} % The references information are stored in the file named "Thesis_bibliography.bib"

%-------------------------------------------------------------------------
%	APPENDICES
%-------------------------------------------------------------------------

\cleardoublepage
\addtocontents{toc}{\vspace{2em}} % Add a gap in the Contents, for aesthetics
\appendix
\chapter{Notation}
If you need to include an appendix to support the research in your thesis, you can place it at the end of the manuscript.
An appendix contains supplementary material (figures, tables, data, codes, mathematical proofs, surveys, \dots)
which supplement the main results contained in the previous chapters.

\begin{table}[H]
\centering
\begin{tabular}{c l} \hline
\textbf{Notation}&\textbf{Description} \\ \hline
$G$&influence graph\\
$V$&set of nodes of $G$\\
$E$&set of edges of $G$\\
$W$&set of influence weights corresponding to each edge in $E$\\
$w_{u,v}$&weight of edge $(u,v)$\\
$n$&$|V|$, number of nodes\\
$m$&$|E|$, number of edges\\
\hline
\end{tabular}
\caption{Notation used.}
\label{tab:thesis_notation}
\end{table}
%\chapter{Running the Simulation}
TODO


% LIST OF FIGURES
\listoffigures

% LIST OF TABLES
\listoftables

% LIST OF SYMBOLS
% Write out the List of Symbols in this page
\chapter*{List of Symbols}
\begin{table}[H]
    \centering
    \begin{tabular}{lll}
        \textbf{Variable} & \textbf{Description} & \textbf{SI unit} \\\hline\\[-9px]
        $\bm{u}$ & solid displacement & m \\[2px]
        $\bm{u}_f$ & fluid displacement & m \\[2px]
    \end{tabular}
\end{table}

% ACKNOWLEDGEMENTS
\chapter*{Acknowledgments}
Here you can insert optionally the acknowledgments for who had a significant importance for the accomplishment of this goal. These acknowledgments are less formal than the ones at the beginning of the thesis and are not listed in the table of contents.

\cleardoublepage

\end{document}
