% The form of the entries in this file is \newacronym{label}{acronym}{phrase}
%                                      or \newacronym[options]{label}{acronym}{phrase}
% see "User Manual for glossaries.sty" for the  details about the options, one example is shown below
% note the specification of the long form plural in the line below
% \newacronym[longplural={Debugging Information Entities}]{DIE}{DIE}{Debugging Information Entity}
%
% The following example also uses options
% \newacronym[plural={OSes}, firstplural={operating systems (OSes)}, first={operating system (OS)}]{OS}{OS}{operating system}

% note the use of a non-breaking dash in long text for the following acronym
% \newacronym{IQL}{IQL}{Independent Q‑Learning}

% \newacronym{LAN}{LAN}{Local Area Network}
% note the use of a non-breaking dash in the following acronym
% \newacronym{WiFi}{Wi-Fi}{Wireless Fidelity}

% \newacronym{WLAN}{WLAN}{Wireless Local Area Network}
% \newacronym{UN}{UN}{United Nations}
% \newacronym{SDG}{SDG}{Sustainable Development Goal}

\newacronym{IoT}{IoT}{Internet-of-Things}

\newacronym[plural={VMs}, firstplural={Virtual Machines (VMs)}, first={Virtual Machine (VM)}]{VM}{VM}{Virtual Machine}

\newacronym{FaaS}{FaaS}{Function-as-a-Service}

\newacronym{API}{API}{Application Programming Interface}

\newacronym{JSON}{JSON}{JavaScript Object Notation}

\newacronym{IDE}{IDE}{Integrated Development Environment}

\newacronym[plural={OSes}, firstplural={Operating Systems (OSes)}, first={Operating System (OS)}]{OS}{OS}{Operating System}


\newacronym{UI}{UI}{User Interface}


% Comment this line if you only want to appear
% the acronyms present in the document
% \glsaddall

% Use acronyms in the text with the following commands
% \gls{ }
% To print the term, lowercase.
%
% \Gls{ }
% The same as \gls but the first letter will be printed in uppercase.
%
% \glspl{ }
% The same as \gls but the term is put in its plural form.
%
% \Glspl{ }
% The same as \Gls but the term is put in its plural form.
%
% \acrlong{ }
% Displays the phrase which the acronyms stands for. Put the label of the acronym inside the braces.
%
% \acrshort{ }
% Prints the acronym whose label is passed as parameter.
%
% \acrfull{ }
% Prints both, the acronym and its definition. 