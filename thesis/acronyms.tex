% The form of the entries in this file is \newacronym{label}{acronym}{phrase}
%                                      or \newacronym[options]{label}{acronym}{phrase}
% see "User Manual for glossaries.sty" for the  details about the options, one example is shown below
% note the specification of the long form plural in the line below
% \newacronym[longplural={Debugging Information Entities}]{DIE}{DIE}{Debugging Information Entity}
%
% The following example also uses options
% \newacronym[plural={OSes}, firstplural={operating systems (OSes)}, first={operating system (OS)}]{OS}{OS}{operating system}

% note the use of a non-breaking dash in long text for the following acronym
% \newacronym{IQL}{IQL}{Independent Q‑Learning}

% \newacronym{LAN}{LAN}{Local Area Network}
% note the use of a non-breaking dash in the following acronym
% \newacronym{WiFi}{Wi-Fi}{Wireless Fidelity}

% \newacronym{WLAN}{WLAN}{Wireless Local Area Network}
% \newacronym{UN}{UN}{United Nations}
% \newacronym{SDG}{SDG}{Sustainable Development Goal}

\newacronym{API}{API}{Application Programming Interface}
\newacronym[longplural={Augmented Realities}, first={Augmented Reality (AR)}]{AR}{AR}{Augmented Reality}

\newacronym{CAD}{CAD}{Computer Aided Design}
\newacronym{DCC}{DCC}{Digital Content Creation}
\newacronym[first={Finite-State Machine (FSM)}, firstplural={Finite-State Machines (FSM)}]{FSM}{FSM}{Finite-State Machine}

\newacronym{GUI}{GUI}{Graphical User Interface}

\newacronym[first={High-Level Content Design Framework (HCDF)}, firstplural={High-Level Content Design Frameworks (HCDFs)}]{HCDF}{HCDF}{High-Level Content Design Framework}
\newacronym[first={Hand Held Device (HHD)}, firstplural={Hand Held Devices (HHDs)}]{HHD}{HHD}{Hand Held Device}
\newacronym[first={Head-Mounted Display (HMD)}, firstplural={Head-Mounted Displays (HMDs)}]{HMD}{HMD}{Head-Mounted Display}

\newacronym{JSON}{JSON}{JavaScript Object Notation}
\newacronym{IDE}{IDE}{Integrated Development Environment}
\newacronym[plural={OSes}, firstplural={operating systems (OSes)}]{OS}{OS}{operating system}
\newacronym[first={Mixed Reality (MR)}]{MR}{MR}{Mixed Reality}
\newacronym{SDK}{SDK}{Software Development Kit}
\newacronym[first={System Usability Score (SUS)}]{SUS}{SUS}{System Usability Score}

\newacronym{UI}{UI}{User Interface}

\newacronym[longplural={Virtual Environments}, first={virtual environment (VE)}]{VE}{VE}{Virtual Environment}
\newacronym[longplural={Virtual Realities}, first={Virtual Reality (VR)}]{VR}{VR}{Virtual Reality}

\newacronym[first={What you see is what you get (WYSIWYG)}]{WYSIWYG}{WYSIWYG}{What You See Is What You Get}
\newacronym[longplural={Extended Realities}, first={Extended Reality (XR)}]{XR}{XR}{Extended Reality}

\newacronym{XRM}{XRM}{Extended Reality Model}

% Comment this line if you only want to appear
% the acronyms present in the document
\glsaddall

% Use acronyms in the text with the following commands
% \gls{ }
% To print the term, lowercase.
%
% \Gls{ }
% The same as \gls but the first letter will be printed in uppercase.
%
% \glspl{ }
% The same as \gls but the term is put in its plural form.
%
% \Glspl{ }
% The same as \Gls but the term is put in its plural form.
%
% \acrlong{ }
% Displays the phrase which the acronyms stands for. Put the label of the acronym inside the braces.
%
% \acrshort{ }
% Prints the acronym whose label is passed as parameter.
%
% \acrfull{ }
% Prints both, the acronym and its definition. 