\section{Conclusions}
\label{sec:conclusions}

The large diffusion of smart devices and IoT sensors has resulted in an \textbf{unprecedented growth} in the amount of collected data. Core-centric approaches have shown to be inefficient as they need to transfer data back and forth between the core and the devices, generating notable latencies. Therefore new approaches, which exploit the \textbf{tremendous power of the edge} of the network, are replacing the core-centric approaches.

In this thesis we have studied the problem of performing \textbf{stateful computations in a geo-distributed and heterogeneous scenario}, that is the edge of the network.

After analyzing the state of the art in the literature, we started \textbf{collecting and organizing the use cases} predominantly affected by bandwidth and latency constraints. With the use cases at hand we \textbf{studied the current frameworks} provided by the industry and we noticed that some of the use cases were left out and couldn't be fulfilled by the available frameworks. This situation forces developers to create ad hoc solutions on the infrastructure, a process which is error-prone and task-specific.

Therefore we tried to solve the gap of fulfillment present in the use cases, by proposing a \textbf{new solution} which supports the characteristics of the use cases left out. We designed and then implemented a \textbf{prototype} for this solution which brings stateful computations and location awareness in contexts where a change of location of the clients does not occur or is not important (the solution in fact does not provide session consistency).

We then \textbf{evaluated} the performance and usability of our prototype in a simple scenario. Instead to evaluate the solution in a complex but more realistic scenario we resorted to a \textbf{discrete-event simulation}.
We found that, by using our framework with the right use cases, we get immense benefits in terms of \textbf{reduced traffic} in the network and in terms of \textbf{lower latencies}, especially in cases where the data aggregation needed is not central. However we also noticed how our solution can be affected by a latency increase due to random spikes in the requests and due to the small number of cores and resources at the edge of the network. Nevertheless the results of the evaluation confirmed the \textbf{power and effectiveness of the proposed solution}.
