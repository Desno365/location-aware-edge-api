\section{Equations, Figures, Tables and Algorithms}
\label{sec:equations_and_figures}
All Figures, Tables and Algorithms have to be properly referred in the text.
Equations have to be numbered only if they are referred in the text.
\subsection{Equations}
\label{sec_equations}
A few important equations related to your work might be reported in the Executive Summary. For example, the Maxwell's equations read:
\begin{subequations}
    \label{eq:maxwell}
    \begin{align}[left=\empheqlbrace]
    \nabla\cdot \bm{D} & = \rho, \label{eq:maxwell1} \\
    \nabla \times \bm{E} +  \frac{\partial \bm{B}}{\partial t} & = \bm{0}, \label{eq:maxwell2} \\
    \nabla\cdot \bm{B} & = 0, \label{eq:maxwell3} \\
    \nabla \times \bm{H} - \frac{\partial \bm{D}}{\partial t} &= \bm{J}. \label{eq:maxwell4}
    \end{align}
\end{subequations}

Equation~\eqref{eq:maxwell} is automatically labeled by \texttt{cleveref},
as well as Equation~\eqref{eq:maxwell1} and Equation~\eqref{eq:maxwell3}.
Thanks to the \verb|cleveref| package, there is no need to use \verb|\eqref|.

\subsection{Figures}
\label{sec:figures}
To include Figures in your text you can use \texttt{TikZ} for high-quality hand-made figures \cite{tikz},
or just include them with the command
\begin{verbatim}
\includegraphics[options]{filename.xxx}
\end{verbatim}
where xxx is the format (\verb|.png|, \verb|.jpg|, \verb|.eps|, \dots).
An example is shown in Figure~\ref{fig:quadtree}.
\begin{figure}[H]
    \centering
    \includegraphics[width=0.3\textwidth]{logo_polimi_scritta.eps}
    \caption{Caption of the Figure.}
    \label{fig:quadtree}
\end{figure}

\subsection{Tables}
\label{subsec:tables}

Within the environments \texttt{table} and  \texttt{tabular} you can create very fancy tables like the one shown in Table~\ref{table:example}.
\begin{table}[H]
    \caption*{\textbf{Example of Table}}
    \centering 
    \begin{tabular}{|p{3em} c c c |}
    \hline
    \rowcolor{bluePoli!40}
     & \textbf{column1} & \textbf{column2} & \textbf{column3} \T\B \\
    \hline \hline
    \textbf{row1} & 1 & 2 & 3 \T\B \\
    \textbf{row2} & $\alpha$ & $\beta$ & $\gamma$ \T\B\\
    \textbf{row3} & alpha & beta & gamma \B\\
    \hline
    \end{tabular}
    \\[10pt]
    \caption{Caption of the Table.}
    \label{table:example}
\end{table}

\subsection{Algorithms}
\label{subsec:algorithms}

Pseudo-algorithms can be written in \LaTeX{} with the \texttt{algorithm} and \texttt{algorithmic} packages.
One example follows.
\begin{algorithm}[H]
\label{alg:example}
\caption{Name of the Algorithm}
\label{alg:var}
\label{protocol1}
\begin{algorithmic}[1]
\STATE Initial instructions
\FOR{$for-condition$}
\STATE{Some instructions}
\IF{$if-condition$}
\STATE{Some other instructions}
\ENDIF
\ENDFOR
\WHILE{$while-condition$}
\STATE{Some further instructions}
\ENDWHILE
\STATE Final instructions
\end{algorithmic}
\end{algorithm} 
